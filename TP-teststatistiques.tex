			\documentclass[11pt,a4paper]{article}
			\usepackage[utf8]{inputenc} 
			\usepackage[T1]{fontenc}
			\usepackage{lmodern}
			\usepackage[french,english]{babel}
			\usepackage{amsmath,amssymb,amsfonts}
			\usepackage[a4paper, margin=1in]{geometry}
			\usepackage{booktabs} % To thicken table lines
			\usepackage{url} % To thicken table lines
			\usepackage[lastexercise]{exercise}
			%\usepackage[lastexercise, answerdelayed]{exercise}
			%\usepackage{answers}
			%\Newassociation{sol}{Solution}{ans}  
			\usepackage{footnote}
			\usepackage{textcomp}
			\makesavenoteenv{tabular}
			\makesavenoteenv{table}
			\usepackage{floatflt}
			\usepackage{rotating}
			\usepackage[]{listings}
			\usepackage{color}
			\usepackage{amsmath}
			\usepackage{multirow}
			\definecolor{bleu-fonce}{rgb}{0,0,0.5}
			%\sectionfont{\color{bleu-fonce}}
			%\geometry{hmargin=2.5cm,vmargin=2cm}
			\definecolor{lightgray}{rgb}{.9,.9,.9}
			\definecolor{darkgray}{rgb}{.4,.4,.4}
			\definecolor{purple}{rgb}{0.65, 0.12, 0.82}
			\def\prox{{\mathbf{prox}}}
			
			\lstset{ %
				language=Matlab,                % the language of the code
				backgroundcolor=\color{lightgray},
				extendedchars=false,
				basicstyle=\footnotesize\ttfamily,
				showstringspaces=false,
				showspaces=false,
				tabsize=2,
				breaklines=true,
				showtabs=false,
				captionpos=b
				keywords={typeof, new, true, false, catch, function, return, null, catch, switch, var, if, in, while, do, else, case, break,ones,endwhile},
				keywordstyle=\color{blue}\bfseries,
				ndkeywords={class, export, boolean, throw, implements, import, this},
				ndkeywordstyle=\color{darkgray}\bfseries,
				identifierstyle=\color{black},
				commentstyle=\color{darkgray},
				upquote = true,
				columns=fullflexible,
			}
			
			
			%-----------------------------------------------------------------
			
			\def\dbR{{\mathrm{I\hskip-2.2pt R}}}
			\def\dbN{{\mathrm{I\hskip-2.2pt N}}}
			\def\esp{{\mathrm{I\hskip-1.5pt E}}}
			\def\pr{{\mathrm{I\hskip-2.2pt P}}}
			\def\halam{{\widehat{\lambda}}}
			\def\hasig{{\widehat{\sigma}}}
			\def\ha{{\widehat{a}}}
			\def\hb{{\widehat{b}}}
			\def\haa{{\widehat{a}}}
			\def\hab{{\widehat{b}}}
			\def\he{{\widehat{e}}}
			\def\haE{{\widehat{E}}}
			\def\baR{{\overline{R}}}
			\def\baE{{\overline{E}}}
			\def\baX{{\overline{X}}}
			\def\baY{{\overline{Y}}}
			\def\lamMV{{\widehat{\lambda}_{MV}}}
			\def\lamef{{\widehat{\lambda}_{e}}}
			
			\def\dbC{{\mathrm{I\hskip-4.7pt C}}}
			\def\0{{\mathrm{0}}}
			\def\a{{\mathrm{a}}}
			\def\b{{\mathrm{b}}}
			\def\c{{\mathrm{c}}}
			\def\d{{\mathrm{d}}}
			\def\e{{\mathrm{e}}}
			\def\f{{\mathrm{f}}}
			\def\u{{\mathrm{u}}}
			\def\v{{\mathrm{v}}}
			\def\w{{\mathrm{w}}}
			\def\x{{\mathrm{x}}}
			\def\y{{\mathrm{y}}}
			\def\z{{\mathrm{z}}}
			\def\clA{{\mathcal{A}}}
			\def\clD{{\mathcal{D}}}
			\def\clH{{\mathcal{H}}}
			\def\clL{{\mathcal{L}}}
			\def\clP{{\mathcal{P}}}
			\def\clO{{\mathcal{O}}}
			\def\tx{{\widetilde{x}}}
			
			\def\Wb{{\overline{W}}}
			
			
			\selectlanguage{english}
			%-----------------------------------------------------------------
			\begin{document}
				
				
				
				
				\title{\vspace*{-2cm}
					\large Tests Statistiques 1}
				\author{L. Noiret, A. Rogozan, M.A Haddouche} 
				\date{ }
				
				\maketitle
			
			
				
				
			
				\section*{Rappels}	
				L'objectif d'un test statistique est de d\'eterminer si une hypoth\`ese $H_0$ doit \^etre rejet\'ee ou non. \\
				Un test comporte deux risques :
				\begin{itemize}
					\item le risque de premi\`ere esp\`ece $\alpha$ qui consiste \` d\'eclarer que $H_0$ est fausse (c'est \`a dire conclure \`a une diff\'erence) alors qu'en r\'ealit\'e $H_0$ est vraie
					\item le risque de deuxi\`eme esp\`ece $\beta$ qui consiste \`a ne pas d\'eclarer que H0 est fausse (conclure qu'il n'y a pas de diff\'erence) alors qu'en r\'ealit\'e $H_0$ est fausse.  
				\end{itemize}
			Dans la pratique  on construit un test en trois étapes: 
				\begin{itemize}
					\item On Formule l'hypoth\`ese que vous voulez tester $H_0$ et l'hypoth\`ese alternative $H_1$. La formulation  unilat\'erale ou bilat\'erale de $H_1$ va d\'eterminer les bornes de la r\'egion de rejet. 
%					
					\item On Choisit le type de test \`a impl\'ementer. Il existe souvent plusieurs tests pour r\'epondre \`a la m\^eme question. Le choix du test est notamment guid\'e par ses conditions d'application.\\ Exemple : si vous souhaitez tester que deux \'echantillons ind\'ependants sont issus de la m\^eme population, vous pouvez utiliser un test de Student  ou un test non param\'etrique tel que Mann Whitney. Comment choisir?
					Vous choississez un test de Student si vos donn\'ees sont gaussiennes ou si la taille de vos 2 \'echantillons est assez grande ($n_1>30$ et $n_2>30$ ), afin que le th\'eor\`eme central limite s'applique. Sinon vous utilisez le test de Mann Whitney.
%					
				\item On interprète les sorties du logiciel : on rejette $H_0$, si la p-value est inf\'erieur au risque de 1\`ere esp\`ece (en g\'en\'eral $\alpha=5\%$). 
					
				\end{itemize}
			
%					\noindent Les tests non param\'etriques ne font aucune hypoth\`eses sur la distribution sous-jacente, mais sont moins puissants que les tests param\'etriques, c'est-\`a-dire qu'ils d\'etectent moins souvent une diff\'erence lorsqu'il y en une. La puissance (1-$\beta$) est la probabilit\'e de d\'eclarer que H0 est fausse \`a tort (conclure qu'il n'y a pas de diff\'erence lorsqu'il y en a une).
			%	Jusqu'`a pr\'esent, lorsque nous avons d\'ecrit des donn\'ees statistiques (analyses descriptive), nous avons utilis\'e notre intuition pour comparer des indicateurs. La moyenne de telle population semble sup\'erieure \`a telle autre, les donn\'ees semblent corr\'el\'ees... Ces conclusions dépendent beaucoup de la personne qui r\'ealise les analyses. Nous allons donc d\'evolopper une approche plus objective pour comparer des populations \`a l'aide de tests statistiques. 
			%	Il existe une myriade de tests statistiques diff\'erents (tests de compraison de moyennes, de variance, de distribution...). Certains font des hypoth\`eses sous jacentes sur la distribution des donn\'ees (par exemple, on suppose tests param\'etriques) alors que d'autres n'en font aucunes (tests non param\'etriques). 
				
			%	Erreur de premi\`ere esp`ece : rejeter l'hypoth`ese H0 alors que H0 est vraie
			%	Puissance ($1-\beta$): accepter H0 lorsque H0 est vraie.
			%	Aujourd'hui, nous allons nous concentr\'er sur les tests param\'etriques.   
			%	Un des tests les plus utilis\'es en statistique est le test de Student qui permet de c%omparer les moyennes de deux populations.\\
			%	H0 : les moyennes des deux populations sont \'egales ($\mu_1=\mu_2$)\\
			%	contre H1  les moyennes sont diff\'erentes ($\mu_1\neq\mu_2$, formulation bilat\'erale), ou l'une des moyennes est sup\'erieure \`a l'autre $\mu_1>\mu_2$ ou $\mu_1<\mu_2$, formulation unilat\'erale).
				
				
				\section*{\Large Application : les cambriolages en France}
%				Ce TP peut se faire sous Matlab, R ou Python (au choix). Des indications sont fournies pour l'utilisation de Python. \\
%				\newline
				On souhaite comparer des donn\'ees statistiques de la d\'elinquance en Ile de France (IDF) avec celles de la r\'egion Provence-Alpes-C\^ote d'Azur (PACA). 
				Les donn\'ees, publi\'ees par le journal l'Express, fournissent le nombre de vols de voitures ($VolVoiture$) et de cambriolages ($Cambriolage$) dans diff\'erentes circonscriptions. Suivant les indicateurs, les d\'elits sont exprim\'es pour 1000 ou 10000 habitants.\\
		\section{Etapes pr\'eliminaires}
			
					- Charger les donn\'ees
%					.et faites une analyse descriptive rapide de la variable $Cambriolage$. 
					\noindent
				%	{\scriptsize
						\begin{lstlisting}
					df=pd.read_csv("DataSecurite3.csv",encoding = "ISO-8859-1", engine='python',delimiter=";")
						\end{lstlisting}
				%	}
%				\begin{enumerate}
%					\item
					 - Cr\'eer 2 nouveaux tableaux : l'un contenant les donn\'ees pour les circonscriptions d'Ile de France (IDF, d\'epartements : 75, 77, 78, 91, 92, 93, 94, 95) et l'autre pour celles de la r\'egion Provine Alpe C\^ote d'Azure (PACA, d\'epartements : 04, 05, 06, 13, 83 , 84)
						\begin{lstlisting}	
						# Example de selection: IDF = df.loc[(df['Dpt'] == '75' | (df['Dpt'] == '77'))]
						\end{lstlisting}
						\begin{enumerate}
							\item Faites une analyse descriptive de la variable cambriolage en des   IDF et PACA. 
				\end{enumerate}	
		\section{Comparaison d'une moyenne \`a une valeur th\'eorique}
			\begin{enumerate}
				\item[2.1] On veut savoir si le  nombre moyen de  cambriolages en IDF est diff\'erent 
				 de la la moyenne nationale
				
				\begin{itemize}
				\item [2.1.1] Formulez les hypothèses $H_0$ et $H_1$.
				\item [2.1.2] Quelle est la loi de la statistique de test sous $H_0$ ?
				\item [2.1.3] Effectuez le test et interpr\^etez la p-value au risque 5\%.
				\begin{lstlisting}
						Utiliser la fonction scipy.stats.ttest_1samp( ,  )
				\end{lstlisting}
%				\item Récupérer la p-value et la statistique du Test
				\end{itemize}
			\item[2.2] On veut savoir si le  nombre moyen de  cambriolages en IDF est sup\'erieur \`a la moyenne 
			nationale
			\begin{itemize}
			\item [2.2.1]Formulez les hypothèses $H_0$ et $H_1$.
		 	\item [2.2.2] Effectuez le test et interpr\^etez la p-value au risque 5\%.
%			\begin{lstlisting}
%			# faire le test et enregistrer les resultats dans la variable res_moy2
%					Utiliser la meme fonction pour le test precedent mais ici la p_value est devise par 2.
%			\end{lstlisting}
			 \item  [2.2.3] On veut identifiez les diff\'erences entre les r\'esultats du test bilat\'eral (dans 2.1) et le test unilat\'eral (dans 2.2). Est-ce que la valeur de la statistique  et le nombre de degr\'es de libert\'es $df$ sont diff\'erents? 
			 	\end{itemize}
		\end{enumerate}
%		\subsection{Calcul de puissance}
%		En utilisant les sorties pr\'ec\'edentes, 
%		
%		power.t.test(n = NULL, delta = NULL, sd = 1, sig.level = 0.05,
%		power = NULL,
%		type = c("two.sample", "one.sample", "paired"),
%		alternative = c("two.sided", "one.sided"),
%		strict = FALSE, tol = .Machine$double.eps^0.25$)
		\section{Comparaison de deux \'echantillons ind\'ependants}				
					On s'int\'eresse au nombre de cambriolages moyen par circonscription. En particulier on veut savoir si les moyennes  dans les r\'egions IDF et  PACA sont-elles statistiquement diff\'erentes
					\begin{itemize}
%						\item Quel test choisiriez-vous? Donner la statistique de test et sa loi sous H0.
						\item[3.1] Avant de proc\'eder au test, nous nous intéressons d'abord aux variances des cambriolages en IDF et PACA sont \'egales (leurs égalité).
						\begin{itemize}	
						\item[3.1.1] Pourquoi a-t-on besoin de tester l'\'egalit\'e des variances?
						\item [3.1.2]Rappelez la statistique de ce test, les hypoth\`eses $H_0$ et $H_1$. Ainsi que la loi de la statistique sous $H_0$ et la zone de rejet .
%						\item[3.1.3] Implémenter le test concluez au risque 5\%
						\end{itemize}
					\begin{itemize}
					\item[3.2] On suppose que les variance sont différentes. Maintenant qu'on a une réponse sur l'égalité des variances, on peut faire le test de comparaison du nombre de cambriolage moyen entre les régions IDF et PACA
					\end{itemize}
					\item [3.2.1] Formulez les hypoth\`eses $H_0$ et $H_1$ permettant de r\'epondre \`a cette question.
		 		 \item [3.2.2] Quel test choisiriez-vous pour la comparaison des moyennes ? Donner la statistique de test et sa loi sous $H_0$.
						\item [3.2.3]R\'ealisez le test de comparaison de moyennes et concluez lorsque le risque $\alpha$ est de 5\%, 1\%.
							\begin{lstlisting}
						Utiliser la fonction stats.ttest_ind( , , )
							\end{lstlisting}
					\end{itemize}
%		\subsection{Comparaison de deux \'echantillons ind\'ependants (formulation unilat\'erale)}			
%				
%					 Le nombre de vols de voiture moyen par circonscription en r\'egion PACA est-il sup\'erieur au nombre de vols en r\'egion Ile de France ?
%					\begin{enumerate}	
%					\item Comment devez-vous adapter H1 pour r\'epondre \`a cette question?
%					\item Regarder les options disponibles de la fonction $t.test$ ($help("t.test")$) pour r\'epondre \`a cette question.
%					\end{enumerate}
%		\subsection{ Calcul de puissance}
%		On note M$_{IDF}$ (respectivement M$_{PACA}$) le nombre moyen de cambriolages en IDF (resp. PACA).
%				\begin{enumerate}	
%					\item Calculez ces valeurs, ainsi que leur diff\'erence M$_{PACA}$-M$_{IDF}$.
%					\item On souhaite tester au risque 5\% les hypoth\`eses suivantes :\\ 
%					H0 M$_{PACA}$-M$_{IDF}$ = 0\\
%					H1 M$_{PACA}$-M$_{IDF}$ = 1.125.\\
%					Pour simplifier, on suppose \'egalement que les variances des deux populations sont \'egales et valent 4.5, et que les \'echantillon sont de m\^eme taille (n=70). 
%					Quelle est la puissance de ce test?
%				\end{enumerate}
%		\subsection{Comparaison de deux \'echantillons ind\'ependants (version non param\'etrique)}			
%		
%		On souhaite maintenant comparer les cambriolages dans les Hauts-de Seine (D\'epartement 92) avec ceux en Seine St Denis (D\'epartement 93). 
%					\begin{enumerate}	
%					\item Cr\'eez 2 nouveaux tableaux : l'un contenant les donn\'ees pour les circonscriptions des Hauts-de Seine $X92$ et l'autre celles de la Seine St Denis $X93$.
%					\item Quels tests choisiriez-vous pour comparer les deux d\'epartements? Pourquoi?
%					\item R\'ealisez ce test. Essayer de comprendre l'origine du message d'avertissement (Warning). R\'ecup\'erez la p-value et concluez.
%				%	\begin{lstlisting}
%%					res_non_para=wilcox.test(X92$Cambriolage, X93$Cambriolage) 
%%					\end{lstlisting}
%					
%				\end{enumerate}
				
			%	\end{enumerate}
		%		
		%				\section{Puissance d'un test}
		%				\subsection{Rappels}
		%	
		%						\subsection{Rappels}
		%	On souhaite comparer deux traitements A et B dans un essai th\'erapeutique sur deux groupes ind\'ependants.
		%	La r\'eponse au traitement est mesur\'ee sur par une variable quantitative X$_A$ (ou X$_B$), de moyenne $\mu_A$ ($\mu_B$) et de variance \'egales $\sigma_A^2=\sigma_B^2=0.5$.
		%	H0 $\mu_A=\mu_B$ (ou $\mu_A-\mu_B=0)$\\
		%	H1 $\mu_A>\mu_B$ (ou $\mu_A-\mu_B>0$)\\ 
		%	On mesure l'effet moyen de chaque traitement sur une population de 50 personnes, on trouve $\bar{x}_n^A=115$ et $\bar{x}_n^B=117$
		%	
		%	Un traitement 
		%	Quelle
			
		%	
		%	 On note $\bar{X}_n^A$ 
		%	On souhaite d\'eterminer si le traitement A est sup\'erieur u traitement B.
		%				On s'int\'eresse au rythme cardiaque.  
		%				Un traitement pour l'hypertension cardiaque est consid\'ere efficace si il permet de diminuer la pression cardiaque de 2 unit\'e (mmHg).
		%				Le traitement de r\'erence 
		
					
				
					
				\section*{Sources} 
				M\'ethode Statistiques M\'edecine - Biologie -Jean Bouyer, ESTEM \'editions Inserm\\
				\newline
				L'Express - S\'ecurit\'e 2013 - Classements des communes de France\\
				\scriptsize{
		 \url{https://lexpress.opendatasoft.com/explore/dataset/statistiques-securite-france-2013/?flg=fr}\\
		\url{http://www.lexpress.fr/actualite/societe/insecurite-le-palmares-des-villes-de-france_1300974.html#m2ZPvIj0jcHs7c1g.99}}
		
				\end{document}
				%%% Local Variables:
				%%% mode: latex
				%%% TeX-master: t
				%%% End:
