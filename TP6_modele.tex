\documentclass{article}

\usepackage[utf8]{inputenc}
\usepackage[T1]{fontenc}
\usepackage[french]{babel}
\usepackage{graphicx}
\usepackage[margin=3cm]{geometry}
\usepackage{fancyhdr}
\usepackage{listings}
\usepackage{amsmath,amssymb,amsfonts,mathrsfs}

\lstset{
    breaklines=true,
    numbers=left,
    frame=single
    }

\pagestyle{fancy}
\fancyhead[L]{NOM(S) et Prénom(s) ---}
\fancyfoot{}

\begin{document}

\begin{center}
    {\Huge Tests statistiques}
\end{center}

% Tout au long de ce TP, pensez à justifier vos réponses !!! N'hésitez pas à rajouter ce que signifie concrètement l'acceptation ou le rejet des hypothèses.

\section{Étapes préliminaires}
\subsection{Analyse descriptive de la variable cambriolage}
Région \underline{IDF} :
\begin{description}
    \item[Effectif :]
    \item[Type :] 
    \item[Moyenne :] 
    \item[Quartile 1 :] 
    \item[Médiane] 
    \item[Quartile 3 :] 
    \item[Min :] 
    \item[Max :] 
\end{description}

Région \underline{PACA} :
\begin{description}
	\item[Effectif :]
	\item[Type :] 
	\item[Moyenne :] 
	\item[Quartile 1 :] 
	\item[Médiane] 
	\item[Quartile 3 :] 
	\item[Min :] 
	\item[Max :] 
\end{description}

\section{Comparaison d'une moyenne à une valeur théorique}

\subsection{Le nombre moyen de cambriolages en IDF est-il différent de la moyenne nationale ?}
\begin{itemize}
	\item[2.1.1] Hypothèses : 
\begin{description}
    \item[H0 :]
    \item[H1 :]
\end{description}
 \item[2.1.2] La statistique de test est .... et sa loi est ...

\item[2.1.3] On trouve une p-valeur de ... : au risque de 5\%, on ... l'hypothèse H0.
\end{itemize}

\subsection{Le nombre moyen de cambriolages en IDF est-il supérieur à la moyenne nationale ?}
\begin{itemize}
	\item[2.2.1] Hypothèses : 
\begin{description}
    \item[H0 :]
    \item[H1 :]
\end{description}

\item[2.2.2] On trouve une p-valeur de ... : au risque de 5\%, on ... l'hypothèse H0.

\item[2.2.4]Quelles sont les différences entre ces deux approches 
\begin{description}
    \item[Nombre de degrés de liberté :]
    \item[Rapport entre les p-valeurs :]
\end{description}
\end{itemize}



\section{Comparaison de deux échantillons indépendants}
\subsection{Test d'égalité de variance de Fisher} 
\begin{itemize}
\item [3.1.1]On a besoin de tester l'égalité des variances parce que ... .

\item [3.1.2] Hypothèses : 
\begin{description}
	\item[H0 :]
	\item[H1 :]
\end{description}
Sous H0, la statistique suit une loi de ...à ... degré de liberté\\
La zone de rejet est....\\
\end{itemize}
\vspace{3mm}


\subsection{Test d'égalité des moyennes }

\begin{itemize}
	\item [3.2.1]
Hypothèses : 
\begin{description}
    \item[H0 :]
    \item[H1 :]
\end{description}
\item[3.2.2]
On choisit le test ... .\\
La statistique de test correspondante est ... .\\
Elle suit une loi de ... sous l'hypothèse H0 à ...dégrée de liberté.\\
\item[3.2.3]
On trouve une p-valeur de ... : au risque de 5\%, on ... l'hypothèse H0.\\
On trouve une p-valeur de ... : au risque de 1\%, on ... l'hypothèse H0.\\
\end{itemize}

\newpage
\appendix

\section{Code}
%\lstinputlisting[language=Matlab]{sourceTP.m}


\end{document}