\documentclass{article}

\usepackage[utf8]{inputenc}
\usepackage[T1]{fontenc}
\usepackage[french]{babel}
\usepackage{graphicx}
\usepackage[margin=3cm]{geometry}
\usepackage{fancyhdr}
\usepackage{listings}
\usepackage{amsmath,amssymb,amsfonts,mathrsfs}

\lstset{
    breaklines=true,
    numbers=left,
    frame=single
    }

\pagestyle{fancy}
\fancyhead[L]{NOM(S) et Prénom(s) ---}
\fancyfoot{}

\begin{document}

\begin{center}
    {\Huge Tests statistiques}
\end{center}

% Tout au long de ce TP, pensez à justifier vos réponses !!! N'hésitez pas à rajouter ce que signifie concrètement l'acceptation ou le rejet des hypothèses.

\section{Étapes préliminaires (0,25 x 2 pts)}
\subsection{Analyse descriptive de la variable Cambriolage}
Analyse descriptive de la variable \underline{IDF} :
\begin{description}
    \item[Effectif : 137.000000]
    \item[Type :] 
    \item[Moyenne : 5.817591] 
    \item[Quartile 1 :] 
    \item[Médiane] 
    \item[Quartile 3 :] 
    \item[Min :] 
    \item[Max :] 
\end{description}

Analyse descriptive de la variable \underline{PACA} :
\begin{description}
	\item[Effectif : 35.000000]
	\item[Type :] 
	\item[Moyenne : 8.300000] 
	\item[Quartile 1 :] 
	\item[Médiane] 
	\item[Quartile 3 :] 
	\item[Min :] 
	\item[Max :] 
\end{description}

\section{Comparaison d'une moyenne à une valeur théorique (1x2 pts)}

\subsection{Le nombre moyen de cambriolages en IDF est-il différent de la moyenne nationale ? (1pts)}
\begin{itemize}
	\item[2.1.1] Hypothèses : 
\begin{description}
    \item[H0 :]  $\mu_{IDF}=5,40$ 
    \item[H1 :]$\mu_{IDF} \neq 5,40$ 
\end{description}
 \item[2.1.2] La statistique de test est $T=\sqrt{n}* \frac{\mu_{IDF} - 5,40}{S}$  et sa loi sous $H_0$ est une  Student $(n-1)$
où $S$ est estimateur sans biais de la variance en IDF

\item[2.1.3] On trouve une p-valeur de 0.0035 : au risque de 5\%, on On rejette $H_0$ .
\end{itemize}

\subsection{Le nombre moyen de cambriolages en IDF est-il supérieur à la moyenne nationale ?(1pts)}
\begin{itemize}
	\item[2.2.1] Hypothèses : 
\begin{description}
    \item[H0 :] $\mu_{IDF}=5,40$
    \item[H1 :] $\mu_{IDF} > 5,40$ 
\end{description}
 
\item[2.2.2] On trouve une p-valeur de 0.00175 : au risque de 5\%, on rejette l'hypothèse H0.

\item[2.2.4]Quelles sont les différences entre ces deux approches 
\begin{description}
    \item[Nombre de degrés de liberté : idem]
    \item[Statisique de test : idem]
\end{description}
\end{itemize}



\section{Comparaison de deux échantillons indépendants (2,5 pts)}
\subsection{Test d'égalité de variance de Fisher (1pts)} 
\begin{itemize}
\item [3.1.1]On a besoin de tester l'égalité des variances parce que la statisque de test change

\item [3.1.2] Hypothèses : 
\begin{description}
	\item[H0 :]  $\sigma^2_{IDF}=\sigma^2_{PACA}$ 
	\item[H1 :] $\sigma^2_{IDF} \neq \sigma^2_{PACA}$
\end{description}
Sous H0, la statistique est  $F = \frac{S^2_{IDF}}{S^2_{PACA}}$ elle suit une loi de  Fisher à  $(n_{IDF}-1,n_{PACA}-1)$ degré liberté \\

\end{itemize}
\vspace{3mm}


\subsection{Test d'égalité des moyennes (1,5 pts)}

\begin{itemize}
	\item [3.2.1]
Hypothèses : 
\begin{description}
    \item[H0 :]
    \item[H1 :]
\end{description}
\item[3.2.2]
On choisit le test Student \\
La statistique de test correspondante est  
$ T = \frac {\bar{x}_{IDF}-\bar{x}_{PACA}}{\sqrt{(S^2_{IDF}/n_{IDF} +S^2_{PACA}/n_{PACA})}} $\\
Elle suit une loi de  Student sous l'hypothèse H0 à ...dégrée de liberté.\\
\item[3.2.3]
On trouve une p-valeur de 5.72643212397114e-06  : au risque de 5\%, on rejette l'hypothèse H0.\\

On trouve une p-valeur de 5.72643212397114e-06 : au risque de 1\%, on rejette l'hypothèse H0.\\
\end{itemize}

\newpage
\appendix

\section{Code}
%\lstinputlisting[language=Matlab]{sourceTP.m}


\end{document}